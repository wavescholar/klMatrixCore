% ------------------------------------------------------------------------
% -*-TeX-*- -*-Hard-*- Smart Wrapping
% ------------------------------------------------------------------------
\def\baselinestretch{1}

\chapter{Systems Engineering \& SCM Aspects of Machine Learning
Software}

\section{Matlab}

\section{R}

\section{BLAS - Atlas \& MKL}

\section{KL Un-managed}

\section{.NET, Data, \& Work-flow}

\section{Visualization}

\subsection{VTK - An OpenGL Rendering Env}

\subsection{Siverlight, WPF, DirectX}



\section{Numerical Libraries}
\subsection{ARPACK}
ARPACK++ is an object-oriented version of the Fortran ARPACK
package. ARPACK is designed to compute a few eigenvalues and
eigenvectors of large scale sparse matrices and pencils via the
Arnoldi process for finding eigenvalues called. These methods
utilize Krylov Subspace Projections for iterative solution that
avoids matrix multiplication.  ARPACK implements the implicit
restarted Arnoldi method which reduces the storage requirements
of the traditional Lanczos iteration for Hermitian matrices and
Arnoldi iteration for general matrices.  The key to the Krylov
method is to calculate the linear subspace of $\Real^{(n,n)}$
induced by span of the first m powers of the image of $b$ under
a linear operator $A$, $\kappa_m(A,b) | A \in \mathbb R^{(n,n)}
b\ in \mathbb R^n = \{b, Ab (A)^2b, \ldots (A)^mb \}$.  This
avoids direct matrix matrix operations when finding the first
few eigenvector, eigenvalue pairs in a large system of linear
equations.

\subsection{ATLAS}
Automatically Tuned Linear Algebra software.
\subsection{IPP, MKL}

\subsection{METIS}
METIS is a software library finite element analysis and graph
partitions.  It also can be used to reduce the fill order of
sparse matrices.

\subsection{CLAPACK}
C Interface to LAPACK.

\subsection{TNT}
Template Numerical Library.

\subsection{UMFPACK}
UMFPACK is a set of routines for solving systems of linear
equations, $Ax = b$, when $A$ is sparse and unsymmetric. It is
based on the Unsymmetric-pattern MultiFrontal method. UMFPACK
factorizes $PAQ$, $PRAQ$ and $PR^{-1}AQ$, into the product
$LU$, where $L$ and $U$ are lower and upper triangular,
respectively, $P$ and $Q$ are permutation matrices, and $R$ is
a diagonal matrix of row scaling factors (or $R = I$ if
row-scaling is not used). Both $P$ and $Q$ are chosen to reduce
fill-in (new nonzeros in $L$ and $U$ that are not present in
$A$). The permutation $P$ has the dual role of reducing fill-in
and maintaining numerical accuracy (via relaxed partial
pivoting and row interchanges). The sparse matrix $A$ can be
square or rectangular, singular or non-singular, and real or
complex (or any combination). Only square matrices $A$ can be
used to solve $Ax = b$ or related systems. Rectangular matrices
can only be factorize

\subsection{SDPA}
SDPA is a software library for solving SDPs using on the
Mehrotra-type predictor-corrector infeasible primal-dual
interior-point method. It is implemented C++ language and
utilizes the machine dependent BLAS such as Intel MKL, ATLAS.
LAPACK routines are used for matrix computations.  Efficient
methods to compute the search directions exploiting the
sparsity of the data matrices are implemented. Sparse or dense
Cholesky factorization for the Schur complemetn matrix is
automatically selected. The calculation of the Schur complement
matrix is implemented in reentrant code. A sparse version of
SDPA is available that uses METIS and SPOOLES libraries for
finding a proper sparse structure of the problem.

\subsection{SPOOLS}
SPOOLES is a library for solving sparse real and complex linear
systems of equations. SPOOLES can factor and solve square
linear systems of equations with symmetric structure, and it
can compute multiple minimum degree, generalized nested
dissection and multisection orderings of matrices with
symmetric structure.  SPOOLES utilizes a variety of Krylov
iterative methods. The preconditioner is a drop tolerance
factorization.


\subsection{AMD}
AMD is a set of routines for pre-ordering a sparse matrix prior
to numerical factorization. It uses an approximate minimum
degree ordering algorithm to find a permutation matrix P so
that the Cholesky factorization $PAP^\dag =LL^\dag$ has fewer
(often much fewer) nonzero entries than the Cholesky
factorization of A. The algorithm is typically much faster than
other ordering methods and minimum degree ordering algorithms
that compute an exact degree . Some methods, such as
approximate deficiency [Rothberg and Eisenstat 1998] and
graph-partitioning based methods [Hendrickson and Rothberg
1999; Karypis and Kumar 1998; Pellegrini et al. 2000; Schulze
2001] can produce better orderings, depending on the matrix.
The algorithm starts with an undirected graph representation of
a symmetric sparse matrix . Node $i$ in the graph corresponds
to row and column i of the matrix, and there is an edge $(i,
j)$ in the graph if $a_{ij}$ is nonzero. The degree of a node
is initialized to the number of off diagonal non-zeros in row
$i$, which is the size of the set of nodes adjacent to $i$ in
the graph.
